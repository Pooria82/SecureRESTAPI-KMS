\documentclass[a4paper,12pt]{article}
\usepackage[utf8]{inputenc}
\usepackage{geometry}
\geometry{margin=1in}
\usepackage{fancyhdr}
\usepackage{setspace}
\usepackage{titlesec}
\usepackage{enumitem}
\usepackage{xepersian}
\settextfont{Amiri}
\setlatintextfont{Times New Roman}
\pagestyle{fancy}
\fancyhf{}
\fancyhead[C]{گزارش فنی: سامانه RESTful API امن}
\fancyfoot[C]{\thepage}
\renewcommand{\headrulewidth}{0.4pt}
\titleformat{\section}{\large\bfseries}{\thesection}{1em}{}
\begin{document}
\onehalfspacing

\section*{گزارش فنی: سامانه RESTful API امن با ذخیره‌سازی رمزنگاری‌شده اطلاعات و مدیریت کلید}

% تشریح جامع معماری کلی سامانه
\section{شرح معماری کلی سامانه}
سامانه مورد نظر یک RESTful API امن است که با هدف ارائه خدمات احراز هویت، مدیریت امن داده‌های حساس و مدیریت کلیدهای رمزنگاری طراحی و پیاده‌سازی شده است. این سامانه با استفاده از چارچوب FastAPI توسعه یافته که به دلیل کارایی بالا، پشتیبانی از برنامه‌نویسی ناهمزمان و تولید مستندات خودکار OpenAPI، به عنوان ابزار اصلی انتخاب شده است. برای ذخیره‌سازی داده‌ها، از پایگاه داده MongoDB استفاده شده که یک پایگاه داده NoSQL انعطاف‌پذیر است و امکان مقیاس‌پذیری افقی و مدیریت داده‌های ساختاریافته و نیمه‌ساختاریافته را فراهم می‌کند. همچنین، مدیریت کلیدهای رمزنگاری به صورت ایزوله و امن با بهره‌گیری از HashiCorp Vault انجام می‌شود که یک ابزار استاندارد صنعتی برای مدیریت اسرار و کلیدها به شمار می‌رود.

سامانه از طریق Docker Compose اجرا می‌شود که استقرار و مدیریت سرویس‌های مختلف را در کانتینرهای جداگانه تسهیل می‌کند و سازگاری بین محیط‌های توسعه و تولید را تضمین می‌کند. معماری سامانه شامل سه بخش اصلی است:
\begin{itemize}
    \item \textbf{برنامه FastAPI}: این بخش مسئول دریافت و پردازش درخواست‌های HTTP، احراز هویت کاربران با استفاده از توکن‌های JWT، رمزنگاری و رمزگشایی داده‌های حساس با کلیدهای اختصاصی هر کاربر و تعامل با پایگاه داده MongoDB و HashiCorp Vault است.
    \item \textbf{پایگاه داده MongoDB}: این پایگاه داده برای ذخیره‌سازی اطلاعات کاربران (مانند نام کاربری، رمز عبور هش‌شده، salt و اطلاعات احراز هویت دو مرحله‌ای) و داده‌های حساس رمزنگاری‌شده (شامل نوع داده و مقدار رمزنگاری‌شده) استفاده می‌شود.
    \item \textbf{HashiCorp Vault}: این ابزار مدیریت امن کلیدهای رمزنگاری را بر عهده دارد و شامل کلید اصلی (Master Key) و کلیدهای منحصر به فرد هر کاربر است که با کلید اصلی رمزنگاری شده‌اند.
\end{itemize}
این معماری تضمین می‌کند که داده‌های حساس در تمامی مراحل انتقال و ذخیره‌سازی رمزنگاری شده باقی بمانند و کلیدهای رمزنگاری به صورت جداگانه و با امنیت بالا مدیریت شوند.

% توضیح دقیق الگوریتم‌های رمزنگاری و هش
\section{الگوریتم‌های رمزنگاری و هش استفاده شده}
برای تأمین امنیت داده‌ها و احراز هویت کاربران، از الگوریتم‌های رمزنگاری و هش استاندارد استفاده شده است:
\begin{itemize}
    \item \textbf{هش رمز عبور}: الگوریتم bcrypt با ضریب کار 12 برای هش کردن رمزهای عبور به کار گرفته شده است که مقاومت بالایی در برابر حملات brute-force ارائه می‌دهد. برای هر کاربر، یک salt منحصر به فرد 16 بایتی با استفاده از تابع \texttt{secrets.token\_hex(16)} تولید می‌شود تا از حملات rainbow table جلوگیری شود. علاوه بر این، یک pepper سراسری که در فایل \texttt{.env} ذخیره شده، به رمز عبور افزوده می‌شود تا لایه امنیتی بیشتری ایجاد کند. خروجی نهایی در فیلد \texttt{hashed\_password} ذخیره می‌شود.
    \item \textbf{رمزنگاری داده‌های حساس}: از الگوریتم AES-256-GCM استفاده شده که یک الگوریتم رمزنگاری متقارن با قابلیت احراز اصالت است. هر کاربر دارای یک کلید 256 بیتی منحصر به فرد است که با کلید اصلی رمزنگاری شده و در Vault ذخیره می‌شود. برای هر عملیات رمزنگاری، یک nonce منحصر به فرد 12 بایتی با \texttt{secrets.token\_bytes(12)} تولید می‌شود تا از تکرار ciphertext جلوگیری شود.
    \item \textbf{امضای توکن JWT}: توکن‌های JWT با الگوریتم HS256 (HMAC-SHA256) امضا می‌شوند و زمان انقضای آن‌ها 1 ساعت تعیین شده است. کلید امضا (\texttt{JWT\_SECRET}) به صورت امن در فایل \texttt{.env} نگهداری می‌شود.
\end{itemize}
این الگوریتم‌ها به دلیل امنیت بالا، استاندارد بودن و پشتیبانی گسترده در کتابخانه‌های رمزنگاری انتخاب شده‌اند و تضمین می‌کنند که رمزهای عبور غیرقابل بازیابی، داده‌های حساس به صورت امن رمزنگاری‌شده و توکن‌های احراز هویت در برابر دستکاری ایمن باشند.

% تشریح کامل ساختار پایگاه داده
\section{ساختار پایگاه داده}
پایگاه داده MongoDB با نام \texttt{secure\_api} شامل دو مجموعه اصلی است که به شرح زیر طراحی شده‌اند:
\begin{itemize}
    \item \textbf{مجموعه \texttt{users}}: این مجموعه شامل اطلاعات کاربران است و فیلدهای زیر را در بر می‌گیرد:
    \begin{itemize}
        \item \texttt{\_id}: شناسه یکتا (ObjectId).
        \item \texttt{username}: نام کاربری منحصر به فرد.
        \item \texttt{email}: آدرس ایمیل برای احراز هویت دو مرحله‌ای.
        \item \texttt{hashed\_password}: هش bcrypt رمز عبور همراه با salt و pepper.
        \item \texttt{salt}: salt منحصر به فرد 16 بایتی.
        \item \texttt{failed\_attempts}: تعداد تلاش‌های ناموفق ورود (حداکثر 5 تلاش).
        \item \texttt{last\_failed\_attempt\_time}: زمان آخرین تلاش ناموفق برای اعمال قفل 15 دقیقه‌ای.
        \item \texttt{totp\_secret}: راز base32 برای TOTP در احراز هویت دو مرحله‌ای.
        \item \texttt{two\_factor\_enabled}: وضعیت فعال بودن احراز هویت دو مرحله‌ای.
    \end{itemize}
    \item \textbf{مجموعه \texttt{sensitive\_data}}: این مجموعه برای ذخیره‌سازی داده‌های حساس رمزنگاری‌شده طراحی شده و شامل فیلدهای زیر است:
    \begin{itemize}
        \item \texttt{\_id}: شناسه یکتا (ObjectId).
        \item \texttt{user\_id}: شناسه کاربر که به \texttt{\_id} در مجموعه \texttt{users} ارجاع می‌دهد.
        \item \texttt{data\_type}: نوع داده حساس (مانند "card\_number").
        \item \texttt{encrypted\_value}: داده رمزنگاری‌شده به صورت hex-encoded (شامل nonce و ciphertext).
    \end{itemize}
\end{itemize}
این ساختار، جداسازی داده‌های احراز هویت و داده‌های حساس را تضمین می‌کند و از ذخیره‌سازی امن و رمزنگاری‌شده داده‌ها اطمینان می‌دهد.

% توضیح جامع روش مدیریت کلید
\section{روش مدیریت کلید}
مدیریت کلیدها با استفاده از HashiCorp Vault به صورت زیر انجام می‌شود:
\begin{itemize}
    \item \textbf{کلید اصلی (Master Key)}: یک کلید AES-256 است که در مسیر \texttt{kv/master\_key} در Vault ذخیره می‌شود. این کلید برای رمزنگاری و رمزگشایی کلیدهای کاربران استفاده می‌شود و در صورت عدم وجود، به صورت خودکار تولید و ذخیره می‌گردد.
    \item \textbf{کلیدهای کاربران}: هر کاربر یک کلید AES-256 منحصر به فرد دارد که در زمان ثبت‌نام تولید شده، با کلید اصلی رمزنگاری می‌شود و همراه با nonce در مسیر \texttt{kv/user\_keys/<username>} در Vault ذخیره می‌گردد.
    \item \textbf{چرخش کلید}: از طریق endpoint \texttt{/rotate-master-key}، فرآیند چرخش کلید اجرا می‌شود که شامل مراحل زیر است:
    \begin{enumerate}
        \item بازیابی کلید اصلی قدیمی.
        \item تولید کلید اصلی جدید.
        \item رمزگشایی کلید هر کاربر با کلید قدیمی و رمزنگاری مجدد آن با کلید جدید.
        \item به‌روزرسانی کلید اصلی در Vault.
    \end{enumerate}
\end{itemize}
این روش، امنیت کلیدها را تضمین کرده و امکان چرخش کلیدها را بدون نیاز به تغییر داده‌های حساس فراهم می‌کند.

% تبیین دلایل انتخاب ابزارها و روش‌ها
\section{دلایل انتخاب ابزارها و روش‌ها}
ابزارها و روش‌های مورد استفاده بر اساس معیارهای زیر انتخاب شده‌اند:
\begin{itemize}
    \item \textbf{FastAPI}: به دلیل کارایی بالا، پشتیبانی از برنامه‌نویسی ناهمزمان، مستندات خودکار OpenAPI و اعتبارسنجی داده‌ها با Pydantic، برای توسعه API امن و مقیاس‌پذیر انتخاب شد.
    \item \textbf{MongoDB}: به عنوان یک پایگاه داده NoSQL، انعطاف‌پذیری در مدیریت داده‌های ساختاریافته و نیمه‌ساختاریافته را فراهم کرده و با کتابخانه \texttt{motor} به صورت ناهمزمان با FastAPI یکپارچه می‌شود.
    \item \textbf{HashiCorp Vault}: به دلیل ارائه امنیت بالا، قابلیت مدیریت دسترسی و چرخش کلید به عنوان یک ابزار استاندارد صنعتی انتخاب شد.
    \item \textbf{Docker Compose}: برای استقرار آسان و مدیریت سرویس‌های چند کانتینری و تضمین سازگاری محیط‌های توسعه و تولید استفاده می‌شود.
    \item \textbf{کتابخانه Cryptography}: به دلیل پیاده‌سازی امن و آزمایش‌شده الگوریتم‌های رمزنگاری مانند AES-256-GCM، برای تضمین امنیت داده‌ها به کار گرفته شد.
\end{itemize}
این ترکیب از ابزارها و روش‌ها، سیستمی امن، مقیاس‌پذیر و قابل نگهداری را فراهم می‌کند که تمامی نیازهای پروژه را برآورده می‌سازد.

\end{document}